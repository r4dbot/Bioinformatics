% Options for packages loaded elsewhere
\PassOptionsToPackage{unicode}{hyperref}
\PassOptionsToPackage{hyphens}{url}
%
\documentclass[
]{article}
\usepackage{amsmath,amssymb}
\usepackage{lmodern}
\usepackage{iftex}
\ifPDFTeX
  \usepackage[T1]{fontenc}
  \usepackage[utf8]{inputenc}
  \usepackage{textcomp} % provide euro and other symbols
\else % if luatex or xetex
  \usepackage{unicode-math}
  \defaultfontfeatures{Scale=MatchLowercase}
  \defaultfontfeatures[\rmfamily]{Ligatures=TeX,Scale=1}
\fi
% Use upquote if available, for straight quotes in verbatim environments
\IfFileExists{upquote.sty}{\usepackage{upquote}}{}
\IfFileExists{microtype.sty}{% use microtype if available
  \usepackage[]{microtype}
  \UseMicrotypeSet[protrusion]{basicmath} % disable protrusion for tt fonts
}{}
\makeatletter
\@ifundefined{KOMAClassName}{% if non-KOMA class
  \IfFileExists{parskip.sty}{%
    \usepackage{parskip}
  }{% else
    \setlength{\parindent}{0pt}
    \setlength{\parskip}{6pt plus 2pt minus 1pt}}
}{% if KOMA class
  \KOMAoptions{parskip=half}}
\makeatother
\usepackage{xcolor}
\IfFileExists{xurl.sty}{\usepackage{xurl}}{} % add URL line breaks if available
\IfFileExists{bookmark.sty}{\usepackage{bookmark}}{\usepackage{hyperref}}
\hypersetup{
  pdftitle={Week\_5\_R\_function\_lab\_1},
  hidelinks,
  pdfcreator={LaTeX via pandoc}}
\urlstyle{same} % disable monospaced font for URLs
\usepackage[margin=1in]{geometry}
\usepackage{color}
\usepackage{fancyvrb}
\newcommand{\VerbBar}{|}
\newcommand{\VERB}{\Verb[commandchars=\\\{\}]}
\DefineVerbatimEnvironment{Highlighting}{Verbatim}{commandchars=\\\{\}}
% Add ',fontsize=\small' for more characters per line
\usepackage{framed}
\definecolor{shadecolor}{RGB}{248,248,248}
\newenvironment{Shaded}{\begin{snugshade}}{\end{snugshade}}
\newcommand{\AlertTok}[1]{\textcolor[rgb]{0.94,0.16,0.16}{#1}}
\newcommand{\AnnotationTok}[1]{\textcolor[rgb]{0.56,0.35,0.01}{\textbf{\textit{#1}}}}
\newcommand{\AttributeTok}[1]{\textcolor[rgb]{0.77,0.63,0.00}{#1}}
\newcommand{\BaseNTok}[1]{\textcolor[rgb]{0.00,0.00,0.81}{#1}}
\newcommand{\BuiltInTok}[1]{#1}
\newcommand{\CharTok}[1]{\textcolor[rgb]{0.31,0.60,0.02}{#1}}
\newcommand{\CommentTok}[1]{\textcolor[rgb]{0.56,0.35,0.01}{\textit{#1}}}
\newcommand{\CommentVarTok}[1]{\textcolor[rgb]{0.56,0.35,0.01}{\textbf{\textit{#1}}}}
\newcommand{\ConstantTok}[1]{\textcolor[rgb]{0.00,0.00,0.00}{#1}}
\newcommand{\ControlFlowTok}[1]{\textcolor[rgb]{0.13,0.29,0.53}{\textbf{#1}}}
\newcommand{\DataTypeTok}[1]{\textcolor[rgb]{0.13,0.29,0.53}{#1}}
\newcommand{\DecValTok}[1]{\textcolor[rgb]{0.00,0.00,0.81}{#1}}
\newcommand{\DocumentationTok}[1]{\textcolor[rgb]{0.56,0.35,0.01}{\textbf{\textit{#1}}}}
\newcommand{\ErrorTok}[1]{\textcolor[rgb]{0.64,0.00,0.00}{\textbf{#1}}}
\newcommand{\ExtensionTok}[1]{#1}
\newcommand{\FloatTok}[1]{\textcolor[rgb]{0.00,0.00,0.81}{#1}}
\newcommand{\FunctionTok}[1]{\textcolor[rgb]{0.00,0.00,0.00}{#1}}
\newcommand{\ImportTok}[1]{#1}
\newcommand{\InformationTok}[1]{\textcolor[rgb]{0.56,0.35,0.01}{\textbf{\textit{#1}}}}
\newcommand{\KeywordTok}[1]{\textcolor[rgb]{0.13,0.29,0.53}{\textbf{#1}}}
\newcommand{\NormalTok}[1]{#1}
\newcommand{\OperatorTok}[1]{\textcolor[rgb]{0.81,0.36,0.00}{\textbf{#1}}}
\newcommand{\OtherTok}[1]{\textcolor[rgb]{0.56,0.35,0.01}{#1}}
\newcommand{\PreprocessorTok}[1]{\textcolor[rgb]{0.56,0.35,0.01}{\textit{#1}}}
\newcommand{\RegionMarkerTok}[1]{#1}
\newcommand{\SpecialCharTok}[1]{\textcolor[rgb]{0.00,0.00,0.00}{#1}}
\newcommand{\SpecialStringTok}[1]{\textcolor[rgb]{0.31,0.60,0.02}{#1}}
\newcommand{\StringTok}[1]{\textcolor[rgb]{0.31,0.60,0.02}{#1}}
\newcommand{\VariableTok}[1]{\textcolor[rgb]{0.00,0.00,0.00}{#1}}
\newcommand{\VerbatimStringTok}[1]{\textcolor[rgb]{0.31,0.60,0.02}{#1}}
\newcommand{\WarningTok}[1]{\textcolor[rgb]{0.56,0.35,0.01}{\textbf{\textit{#1}}}}
\usepackage{graphicx}
\makeatletter
\def\maxwidth{\ifdim\Gin@nat@width>\linewidth\linewidth\else\Gin@nat@width\fi}
\def\maxheight{\ifdim\Gin@nat@height>\textheight\textheight\else\Gin@nat@height\fi}
\makeatother
% Scale images if necessary, so that they will not overflow the page
% margins by default, and it is still possible to overwrite the defaults
% using explicit options in \includegraphics[width, height, ...]{}
\setkeys{Gin}{width=\maxwidth,height=\maxheight,keepaspectratio}
% Set default figure placement to htbp
\makeatletter
\def\fps@figure{htbp}
\makeatother
\setlength{\emergencystretch}{3em} % prevent overfull lines
\providecommand{\tightlist}{%
  \setlength{\itemsep}{0pt}\setlength{\parskip}{0pt}}
\setcounter{secnumdepth}{-\maxdimen} % remove section numbering
\ifLuaTeX
  \usepackage{selnolig}  % disable illegal ligatures
\fi

\title{Week\_5\_R\_function\_lab\_1}
\author{}
\date{\vspace{-2.5em}}

\begin{document}
\maketitle

\textbf{Q1}. Write a function \textbf{grade()} to determine an overall
grade from a vector of student homework assignment scores dropping the
lowest single score. If a student misses a homework (i.e.~has an
\textbf{NA} value) this can be used as a score to be potentially
dropped. Your final function should be adquately explained with code
comments and be able to work on an example class gradebook such as this
one in CSV format: ``\url{https://tinyurl.com/gradeinput}''
{[}\textbf{3pts}{]}

\hypertarget{example-input-vectors-to-start-with}{%
\subsubsection{Example input vectors to start
with}\label{example-input-vectors-to-start-with}}

\begin{Shaded}
\begin{Highlighting}[]
\CommentTok{\# Example input vectors to start with}
\NormalTok{student1 }\OtherTok{\textless{}{-}} \FunctionTok{c}\NormalTok{(}\DecValTok{100}\NormalTok{, }\DecValTok{100}\NormalTok{, }\DecValTok{100}\NormalTok{, }\DecValTok{100}\NormalTok{, }\DecValTok{100}\NormalTok{, }\DecValTok{100}\NormalTok{, }\DecValTok{100}\NormalTok{, }\DecValTok{90}\NormalTok{)}
\NormalTok{student2 }\OtherTok{\textless{}{-}} \FunctionTok{c}\NormalTok{(}\DecValTok{100}\NormalTok{, }\ConstantTok{NA}\NormalTok{, }\DecValTok{90}\NormalTok{, }\DecValTok{90}\NormalTok{, }\DecValTok{90}\NormalTok{, }\DecValTok{90}\NormalTok{, }\DecValTok{97}\NormalTok{, }\DecValTok{80}\NormalTok{)}
\NormalTok{student3 }\OtherTok{\textless{}{-}} \FunctionTok{c}\NormalTok{(}\DecValTok{90}\NormalTok{, }\ConstantTok{NA}\NormalTok{, }\ConstantTok{NA}\NormalTok{, }\ConstantTok{NA}\NormalTok{, }\ConstantTok{NA}\NormalTok{, }\ConstantTok{NA}\NormalTok{, }\ConstantTok{NA}\NormalTok{, }\ConstantTok{NA}\NormalTok{)}
\end{Highlighting}
\end{Shaded}

\begin{Shaded}
\begin{Highlighting}[]
\NormalTok{student1}
\end{Highlighting}
\end{Shaded}

\begin{verbatim}
## [1] 100 100 100 100 100 100 100  90
\end{verbatim}

\begin{Shaded}
\begin{Highlighting}[]
\FunctionTok{mean}\NormalTok{(student1)}
\end{Highlighting}
\end{Shaded}

\begin{verbatim}
## [1] 98.75
\end{verbatim}

\begin{Shaded}
\begin{Highlighting}[]
\FunctionTok{min}\NormalTok{(student1)}
\end{Highlighting}
\end{Shaded}

\begin{verbatim}
## [1] 90
\end{verbatim}

To find the position of the smallest value use the \texttt{which.min()}

\begin{Shaded}
\begin{Highlighting}[]
\FunctionTok{which.min}\NormalTok{(student1)}
\end{Highlighting}
\end{Shaded}

\begin{verbatim}
## [1] 8
\end{verbatim}

\begin{Shaded}
\begin{Highlighting}[]
\NormalTok{student1[}\FunctionTok{which.min}\NormalTok{(student1)]}
\end{Highlighting}
\end{Shaded}

\begin{verbatim}
## [1] 90
\end{verbatim}

to get everything except the min value

\begin{Shaded}
\begin{Highlighting}[]
\NormalTok{student1[}\SpecialCharTok{{-}}\FunctionTok{which.min}\NormalTok{(student1)]}
\end{Highlighting}
\end{Shaded}

\begin{verbatim}
## [1] 100 100 100 100 100 100 100
\end{verbatim}

\begin{Shaded}
\begin{Highlighting}[]
\FunctionTok{mean}\NormalTok{(student1[}\SpecialCharTok{{-}}\FunctionTok{which.min}\NormalTok{(student1)])}
\end{Highlighting}
\end{Shaded}

\begin{verbatim}
## [1] 100
\end{verbatim}

\begin{Shaded}
\begin{Highlighting}[]
\NormalTok{student2}
\end{Highlighting}
\end{Shaded}

\begin{verbatim}
## [1] 100  NA  90  90  90  90  97  80
\end{verbatim}

\begin{Shaded}
\begin{Highlighting}[]
\FunctionTok{mean}\NormalTok{(student2, }\AttributeTok{na.rm =} \ConstantTok{TRUE}\NormalTok{)}
\end{Highlighting}
\end{Shaded}

\begin{verbatim}
## [1] 91
\end{verbatim}

\begin{Shaded}
\begin{Highlighting}[]
\FunctionTok{mean}\NormalTok{(student2[}\SpecialCharTok{{-}}\DecValTok{2}\NormalTok{])}
\end{Highlighting}
\end{Shaded}

\begin{verbatim}
## [1] 91
\end{verbatim}

\begin{Shaded}
\begin{Highlighting}[]
\NormalTok{student3}
\end{Highlighting}
\end{Shaded}

\begin{verbatim}
## [1] 90 NA NA NA NA NA NA NA
\end{verbatim}

\begin{Shaded}
\begin{Highlighting}[]
\FunctionTok{mean}\NormalTok{(student3,}\AttributeTok{na.rm =}\NormalTok{ T)}
\end{Highlighting}
\end{Shaded}

\begin{verbatim}
## [1] 90
\end{verbatim}

\begin{Shaded}
\begin{Highlighting}[]
\FunctionTok{mean}\NormalTok{(student3)}
\end{Highlighting}
\end{Shaded}

\begin{verbatim}
## [1] NA
\end{verbatim}

Change the NA values to zero use \texttt{is.na()} function to identify
NA value in the vector

\begin{Shaded}
\begin{Highlighting}[]
\FunctionTok{is.na}\NormalTok{(student2)}
\end{Highlighting}
\end{Shaded}

\begin{verbatim}
## [1] FALSE  TRUE FALSE FALSE FALSE FALSE FALSE FALSE
\end{verbatim}

\begin{Shaded}
\begin{Highlighting}[]
\NormalTok{student2[}\FunctionTok{is.na}\NormalTok{(student2)]}
\end{Highlighting}
\end{Shaded}

\begin{verbatim}
## [1] NA
\end{verbatim}

\begin{Shaded}
\begin{Highlighting}[]
\NormalTok{x}\OtherTok{\textless{}{-}}\NormalTok{ student3}
\NormalTok{student3[}\FunctionTok{is.na}\NormalTok{(student3)]}\OtherTok{\textless{}{-}}\DecValTok{0}
\NormalTok{x}
\end{Highlighting}
\end{Shaded}

\begin{verbatim}
## [1] 90 NA NA NA NA NA NA NA
\end{verbatim}

\begin{Shaded}
\begin{Highlighting}[]
\FunctionTok{mean}\NormalTok{(x)}
\end{Highlighting}
\end{Shaded}

\begin{verbatim}
## [1] NA
\end{verbatim}

\begin{Shaded}
\begin{Highlighting}[]
\FunctionTok{mean}\NormalTok{(x[}\SpecialCharTok{{-}}\FunctionTok{which.min}\NormalTok{(x)])}
\end{Highlighting}
\end{Shaded}

\begin{verbatim}
## [1] NA
\end{verbatim}

\begin{Shaded}
\begin{Highlighting}[]
\NormalTok{x[}\FunctionTok{which.min}\NormalTok{(x)]}
\end{Highlighting}
\end{Shaded}

\begin{verbatim}
## [1] 90
\end{verbatim}

\begin{Shaded}
\begin{Highlighting}[]
\FunctionTok{which.min}\NormalTok{(x)}
\end{Highlighting}
\end{Shaded}

\begin{verbatim}
## [1] 1
\end{verbatim}

\hypertarget{making-the-function}{%
\subsubsection{Making the function}\label{making-the-function}}

\begin{Shaded}
\begin{Highlighting}[]
\CommentTok{\#\textquotesingle{} Title}
\CommentTok{\#\textquotesingle{}}
\CommentTok{\#\textquotesingle{} @param x }
\CommentTok{\#\textquotesingle{}}
\CommentTok{\#\textquotesingle{} @return}
\CommentTok{\#\textquotesingle{} @export}
\CommentTok{\#\textquotesingle{}}
\CommentTok{\#\textquotesingle{} @examples}
\NormalTok{grade }\OtherTok{\textless{}{-}} \ControlFlowTok{function}\NormalTok{(x)\{}
\CommentTok{\# map NA missing homework values to zero }
\CommentTok{\# missing homework scores zero}
\NormalTok{  x[}\FunctionTok{is.na}\NormalTok{(x)] }\OtherTok{\textless{}{-}}\DecValTok{0}
\CommentTok{\#  print(x)}
\CommentTok{\# print(mean(x)) }
  \CommentTok{\# excluding the lowest value score homework}
  \FunctionTok{print}\NormalTok{(}\FunctionTok{mean}\NormalTok{(x[}\SpecialCharTok{{-}}\FunctionTok{which.min}\NormalTok{(x)]))}
\NormalTok{\}}
\end{Highlighting}
\end{Shaded}

\begin{Shaded}
\begin{Highlighting}[]
\FunctionTok{grade}\NormalTok{(student3)}
\end{Highlighting}
\end{Shaded}

\begin{verbatim}
## [1] 12.85714
\end{verbatim}

\begin{Shaded}
\begin{Highlighting}[]
\FunctionTok{grade}\NormalTok{(student2)}
\end{Highlighting}
\end{Shaded}

\begin{verbatim}
## [1] 91
\end{verbatim}

\begin{Shaded}
\begin{Highlighting}[]
\FunctionTok{grade}\NormalTok{(student1)}
\end{Highlighting}
\end{Shaded}

\begin{verbatim}
## [1] 100
\end{verbatim}

\begin{Shaded}
\begin{Highlighting}[]
\NormalTok{gradebook }\OtherTok{=} \FunctionTok{read.csv}\NormalTok{(}\AttributeTok{file =}\StringTok{"C:/Users/User/Bioinformatics/BGGN{-}213/Week\_5/R\_week5/class06/student\_homework (1).csv"}\NormalTok{, }\AttributeTok{row.names =} \DecValTok{1}\NormalTok{)}
\CommentTok{\# data}
\CommentTok{\# row.names(data)}
\end{Highlighting}
\end{Shaded}

\begin{Shaded}
\begin{Highlighting}[]
\FunctionTok{apply}\NormalTok{(gradebook,}\DecValTok{1}\NormalTok{,grade)}
\end{Highlighting}
\end{Shaded}

\begin{verbatim}
## [1] 91.75
## [1] 82.5
## [1] 84.25
## [1] 84.25
## [1] 88.25
## [1] 89
## [1] 94
## [1] 93.75
## [1] 87.75
## [1] 79
## [1] 86
## [1] 91.75
## [1] 92.25
## [1] 87.75
## [1] 78.75
## [1] 89.5
## [1] 88
## [1] 94.5
## [1] 82.75
## [1] 82.75
\end{verbatim}

\begin{verbatim}
##  student-1  student-2  student-3  student-4  student-5  student-6  student-7 
##      91.75      82.50      84.25      84.25      88.25      89.00      94.00 
##  student-8  student-9 student-10 student-11 student-12 student-13 student-14 
##      93.75      87.75      79.00      86.00      91.75      92.25      87.75 
## student-15 student-16 student-17 student-18 student-19 student-20 
##      78.75      89.50      88.00      94.50      82.75      82.75
\end{verbatim}

\begin{Shaded}
\begin{Highlighting}[]
\NormalTok{results}\OtherTok{\textless{}{-}}\FunctionTok{apply}\NormalTok{(gradebook, }\DecValTok{1}\NormalTok{, grade)}
\end{Highlighting}
\end{Shaded}

\begin{verbatim}
## [1] 91.75
## [1] 82.5
## [1] 84.25
## [1] 84.25
## [1] 88.25
## [1] 89
## [1] 94
## [1] 93.75
## [1] 87.75
## [1] 79
## [1] 86
## [1] 91.75
## [1] 92.25
## [1] 87.75
## [1] 78.75
## [1] 89.5
## [1] 88
## [1] 94.5
## [1] 82.75
## [1] 82.75
\end{verbatim}

\begin{Shaded}
\begin{Highlighting}[]
\FunctionTok{sort}\NormalTok{(results)}
\end{Highlighting}
\end{Shaded}

\begin{verbatim}
## student-15 student-10  student-2 student-19 student-20  student-3  student-4 
##      78.75      79.00      82.50      82.75      82.75      84.25      84.25 
## student-11  student-9 student-14 student-17  student-5  student-6 student-16 
##      86.00      87.75      87.75      88.00      88.25      89.00      89.50 
##  student-1 student-12 student-13  student-8  student-7 student-18 
##      91.75      91.75      92.25      93.75      94.00      94.50
\end{verbatim}

\begin{Shaded}
\begin{Highlighting}[]
\FunctionTok{which.max}\NormalTok{(results) }
\end{Highlighting}
\end{Shaded}

\begin{verbatim}
## student-18 
##         18
\end{verbatim}

\begin{Shaded}
\begin{Highlighting}[]
\NormalTok{hw.ave}\OtherTok{\textless{}{-}}\FunctionTok{apply}\NormalTok{(gradebook,}\DecValTok{2}\NormalTok{,mean, }\AttributeTok{na.rm=}\ConstantTok{TRUE}\NormalTok{)}
\FunctionTok{which.min}\NormalTok{(hw.ave)}
\end{Highlighting}
\end{Shaded}

\begin{verbatim}
## hw3 
##   3
\end{verbatim}

\begin{Shaded}
\begin{Highlighting}[]
\NormalTok{hw.ave}
\end{Highlighting}
\end{Shaded}

\begin{verbatim}
##      hw1      hw2      hw3      hw4      hw5 
## 89.00000 80.88889 80.80000 89.63158 83.42105
\end{verbatim}

\begin{Shaded}
\begin{Highlighting}[]
\NormalTok{hw.med }\OtherTok{\textless{}{-}} \FunctionTok{apply}\NormalTok{(gradebook, }\DecValTok{2}\NormalTok{, median, }\AttributeTok{na.rm=}\NormalTok{T)}
\FunctionTok{which.min}\NormalTok{(hw.med)}
\end{Highlighting}
\end{Shaded}

\begin{verbatim}
## hw2 
##   2
\end{verbatim}

\begin{Shaded}
\begin{Highlighting}[]
\NormalTok{hw.med}
\end{Highlighting}
\end{Shaded}

\begin{verbatim}
##  hw1  hw2  hw3  hw4  hw5 
## 89.0 72.5 76.5 88.0 78.0
\end{verbatim}

Plot the data using \texttt{boxplot()}

\begin{Shaded}
\begin{Highlighting}[]
\FunctionTok{boxplot}\NormalTok{(gradebook)}
\end{Highlighting}
\end{Shaded}

\includegraphics{week_5_lab_1_files/figure-latex/unnamed-chunk-16-1.pdf}

\begin{Shaded}
\begin{Highlighting}[]
\NormalTok{results}
\end{Highlighting}
\end{Shaded}

\begin{verbatim}
##  student-1  student-2  student-3  student-4  student-5  student-6  student-7 
##      91.75      82.50      84.25      84.25      88.25      89.00      94.00 
##  student-8  student-9 student-10 student-11 student-12 student-13 student-14 
##      93.75      87.75      79.00      86.00      91.75      92.25      87.75 
## student-15 student-16 student-17 student-18 student-19 student-20 
##      78.75      89.50      88.00      94.50      82.75      82.75
\end{verbatim}

\begin{Shaded}
\begin{Highlighting}[]
\FunctionTok{cor}\NormalTok{(results,gradebook)}
\end{Highlighting}
\end{Shaded}

\begin{verbatim}
##            hw1 hw2       hw3 hw4 hw5
## [1,] 0.4250204  NA 0.3042561  NA  NA
\end{verbatim}

\begin{Shaded}
\begin{Highlighting}[]
\NormalTok{gradebook[}\FunctionTok{is.na}\NormalTok{(gradebook)]}\OtherTok{\textless{}{-}}\DecValTok{0}
\NormalTok{gradebook}
\end{Highlighting}
\end{Shaded}

\begin{verbatim}
##            hw1 hw2 hw3 hw4 hw5
## student-1  100  73 100  88  79
## student-2   85  64  78  89  78
## student-3   83  69  77 100  77
## student-4   88   0  73 100  76
## student-5   88 100  75  86  79
## student-6   89  78 100  89  77
## student-7   89 100  74  87 100
## student-8   89 100  76  86 100
## student-9   86 100  77  88  77
## student-10  89  72  79   0  76
## student-11  82  66  78  84 100
## student-12 100  70  75  92 100
## student-13  89 100  76 100  80
## student-14  85 100  77  89  76
## student-15  85  65  76  89   0
## student-16  92 100  74  89  77
## student-17  88  63 100  86  78
## student-18  91   0 100  87 100
## student-19  91  68  75  86  79
## student-20  91  68  76  88  76
\end{verbatim}

\begin{Shaded}
\begin{Highlighting}[]
\FunctionTok{apply}\NormalTok{(gradebook,}\DecValTok{2}\NormalTok{,cor, }\AttributeTok{x=}\NormalTok{results)}
\end{Highlighting}
\end{Shaded}

\begin{verbatim}
##       hw1       hw2       hw3       hw4       hw5 
## 0.4250204 0.1767780 0.3042561 0.3810884 0.6325982
\end{verbatim}

\textbf{Q2.} Using your \textbf{grade()} function and the supplied
\href{https://tinyurl.com/gradeinput}{gradebook}, Who is the top scoring
student overall in the gradebook? {[}\textbf{3pts}{]}

\textbf{Q3.} From your analysis of the gradebook, which homework was
toughest on students (i.e.~obtained the lowest scores overall?
{[}\textbf{2pts}{]}

\textbf{Q4.} \emph{Optional Extension:} From your analysis of the
gradebook, which homework was most predictive of overall score
(i.e.~highest correlation with average grade score)? {[}\textbf{1pt}{]}

\textbf{Q5.} Make sure you save your Rmarkdown document and can click
the \textbf{``Knit''} button to generate a PDF foramt report without
errors. Finally, submit your PDF to gradescope. {[}\textbf{1pt}{]}

\end{document}
